 
\documentclass[10pt]{res} % default is 10 pt
% margin option puts section titles to left of text
%\usepackage{helvetica} % uses helvetica postscript font (download helvetica.sty)
%\usepackage{newcent}   % uses new century schoolbook postscript font 
\usepackage{enumitem}
\usepackage{blindtext}
\usepackage{hyperref}
\topmargin=-0.25in
\addtolength{\textwidth}{.51in}
\addtolength{\textheight}{.51in}
\addtolength{\oddsidemargin}{-.51in}
\begin{document}
\name{SHAYNE M. PLOURDE (He/Him/His)\\[12pt]} % the \\[12pt] adds a blank line after name

\address{{\bf Address} \\  630 Biological Sciences Building\\484 W 12th Ave  \\ Columbus, OH 43210}
\address{ shayne.plourde@outlook.com \\  \url{www.linkedin.com/in/shayne-plourde-a9aa38189} \\ \url{https://shayne-falco.github.io/}
         }

\begin{resume}

\section{OVERVIEW}
    I am a PhD candidate with over 8 years of research experience in Mathematics, Computer Science, and Biology. I possess the skills to develop mathematical models and informative data visualizations for data that I collect myself or receive from collaborations, which are strengthened by my biological knowledge. I am looking for a career that allows me to bridge the gap between biological experiments and mathematical modeling that applies my large collection of skills in both fields. I have enjoyed my time as a bench scientist but look forward to being more on the computational side designing and running computer simulations.
 
\section{EDUCATION}


        PhD Molecular, Cellular, and Developmental Biology\\
        The Ohio State University, Columbus, OH\\
        GPA: 3.42/4.0 \\
        Expected Graduation: Fall, 2023 \\
        Dissertation {\it in silico and in vivo exploration of centrosome maturation factors in early Caenorhabditis elegans development}
                
        M.M.S. Mathematical Biosciences\\
        The Ohio State University, Columbus, OH\\
        GPA: 3.54/4.0 \\
        %Graduated: May, 2017\\
        Thesis {\it Modelling Pollen Aperture Formation with a System of Partial Differential Equations and Turing Dynamics}
                
        B.A., Mathematics \\
        Computer Science Minor\\
        The University of Maine, Orono ME \\ 
        GPA: 3.52/4.0 \\
        %Graduated: May, 2015 \\
         Graduated Magna Cum Laude and with High Honors: May, 2015\\
        Honors Thesis {\it Computational Growth Model of Breast Microcalcification Clusters in Simulated Mammographic Environments}\\
 
 
\section{EXPERIENCE}      
		{\bf Graduate Research Associate} \\
        \begin{ncolumn}{2} % produces two equally spaced columns
        The Ohio State University, Columbus, OH     &      May 2017-present 
        \end{ncolumn}
                
        Worked on a wide variety of research projects involving different model organisms, including \textit{Caenorhabditis elegans}, \textit{Hydra vulgaris}, and \textit{Arabidopsis thaliana}. My research focus was on characterizing the biological mechanisms responsible for pattern formation or positioning of cellular components. This research included designing and undertaking novel experiments and developing various types of mathematical models to recapitulate the processes required to generate the patterns or positioning.
                
        {\bf Laboratory Leader in Biology} \\
        \begin{ncolumn}{2} % produces two equally spaced columns
        The Ohio State University, Columbus, OH     &      Fall 2018 \& 2019 
        \end{ncolumn}

        Teaching associate in Bio 1101 and Bio 1110. Duties included supervising two laboratory sections, grading exams and lab reports, creating quizzes, tutoring, holding office hours. I was solely responsible for running an introductory biology lab designed to teach Freshmen basic biological techniques. BIO 1101 included an Course-based Undergraduate Research Experience (CURE) component giving real world laboratory experience investigating antibacterial resistance in the environment.
                
		{\bf Recitation Instructor in Mathematics} \\
        \begin{ncolumn}{2} % produces two equally spaced columns
        The Ohio State University, Columbus, OH     &      Sept/2015-May 2017 
        \end{ncolumn}

        Teaching associate in Math 1151, Calculus I. Duties included running two recitation sections, grading exams and quizzes, creating quizzes, tutoring, holding office hours. During this time I gained experience working on an interdisciplinary mathematical biology research project that that became my thesis.
 
        {\bf Reserch Assistant} \\ 
        \begin{ncolumn}{2} % produces two equally spaced columns
        CompuMAINE lab, Orono, ME &   Nov 2013-Aug 2015 
        \end{ncolumn}

        I also worked on the research project of an interdisciplinary PhD student. Experience working on an extended project over the course of two years. Applied for and was awarded funding to pursue this research over a summer. Research lead to publication of results.
                
        {\bf MAT 332, Stats for Engineers Grader}\\
        \begin{ncolumn}{2}
        The University of Maine, Orono, ME  &  Sept 2013-Dec 2013
        \end{ncolumn}
                
        Graded tests, organized all class grades and reported scores to students online.

\section{PROJECTS}
    {\bf Mathematical and Biological Exploration of Cellular Patterning Based on Ground Truths}

    I developed an unbiased ODE model of centrosome maturation. This model was found to capture the dynamics of three biological hypotheses, giving them a novel mathematical understanding.
    I acquired and analyzed over 100 high-quality microscopy images to parameterize models.

    
    {\bf Detecting Fake News: A Machine Learning \& Data Science Approach in Python}

    I contributed to the development of a machine learning model to     predict the truthfulness of news. 
    We discovered features like the author and source were over correlated and showed that when such features were removed, the model was still able to achieve an accuracy of 67\%.

    {\bf Identifying Novel Pollen Patterning Mutants in silico, Prior to Biological Experiments}

    I developed a Turing model that recapitulated the pollen surface patterning of Arabidopsis thaliana, and predicted the behavior of two novel mutants before in vivo experiments. 
    I collaborated with a team of four biologists to improve the modeling and biological experiments.
    
    {\bf Agent Based Model of Microcalcification Growth Reveals Importance of Tissue Composition}

    Built an agent based model of tumor growth in simulated microenvironments of different compositions. Discovered that the composition of the fatty and dense tissue in the microenvironment impacts the growth and the chances of metastasis as determined by fractal dimension.
 
\section{FUNDING AWARDS}        
    Rhodus Student Research Assistant position \\%- Summer 2016\\
    CLAS Undergraduate Research and Creative Activity Fellowship %- Summer 2015

\section{HONORS \& SCHOLARSHIPS}
    Asked to judge faculty posters at the 2023 international SMB conference\\
    Dean's List \\
    Pi Mu Epsilon  Induction\\
    George \& Helen Weston Scholarship\\
    Dean John E. Stewart Memorial Scholarship\\
    UMaine Merit Scholarship
                
 
\section{SCIENTIFIC TALKS}
    \begin{itemize}[leftmargin=0pt]
        \item[] Society for Mathematical Biology Invited Speaker - July 2023
        \begin{itemize}
            \item[] {\it Asymmetric Centrosome Maturation in the Early C. elegans Embryo: Insights from Multi-scale Microscopy and Modeling}
        \end{itemize}
        \item[] Ohio Area Worm Meeting - Spring 2023
        \begin{itemize}
            \item[] {\it Asymmetric Centrosomes in Early C. elegans Development}
        \end{itemize}
        \item[] Interdisciplinary Graduate Program Annual Symposium Selected Talk - Spring 2022 \& 2023 
        \begin{itemize}
            \item[] {\it in vivo and in silico Exploration of Asymmetric Centrosomes in the Early C. elegans Embryo}
        \end{itemize}
        \item[] UMaine Undergraduate Research Symposium - Summer 2015
        \begin{itemize}
            \item[] {\it Computational Growth Model of Breast Microcalcification Clusters in Simulated Mammographic Environments}
        \end{itemize}
\end{itemize}
            
\section{POSTERS}
    Interdisciplinary Graduate Program Poster Presentation -  Spring 2021
    \begin{itemize}
            \item[] {\it in vivo and in silico Exploration of Asymmetric Centrosomes in the Early C. elegans Embryo}
        \end{itemize}            
\section{CONFERENCES \& WORKSHOPS }
    Mechanics of Life 2: Models and Methods workshop - December 2023
    Society for Mathematical Biology Annual Meeting - July 2023\\
    Bridging multiscale modeling and practical clinical applications in infectious diseases - July 2023\\
    CURE TALC 2018 Fall Teaching Associate Workshop for CURE Laboratory Instructors\\
	NIMBioS/MBI 2017 Summer Workshop - Connecting Biological Data with Mathematical Models

\section{CERTIFICATES}
    The Erdos Institute 2023 DATA SCIENCE BOOT CAMP\\
    {\bf Final Group Project} Detecting Fake News: A Machine Learning Approach\\
    The spread of misinformation, or "fake news," is a significant issue in today's digital society. This project aims to tackle this challenge by developing a machine-learning model capable of analyzing news articles and classifying them as "real" or "fake."\\
    \url{https://www.erdosinstitute.org/certificates/spring-2023/data-science-boot-camp/shayne-plourde}
            
\section{SKILLS}
{\bf Programming Languages \& Related}\\
    Python, Matlab, Julia, R, Inkscape, XMGrace, LaTeX, Pandas, Matplotlib\\
    Troubleshooting including how to google questions, documentation of code and processes, data analysis pipelines and visualizations, parameterization techniques, Differential equations (ODE/PDEs), Object Oriented Programming (OOP), version control with GitHub

{\bf Biological / Laboratory}\\
    Purchasing supplies within our budget, sterile technique for making all reagents and media used in the lab, fluorescent microscopy techniques, experimental design that is informative to mathematical models, scientific writing and had the pleasure of mentoring 5 undergraduate and graduate students.

{\bf Other Skills}\\
    Collaboration with mathematicians and biologists/other scientists, public speaking, Bioinformatics, independent research.\\    
    I also learned how to handle emergency situations. Among the worst was when the lab lost power and was flooded on December 24th, 2022. We were without power for three weeks and all students were set back by at least one month. During the initial hours of the emergency, I was able to prevent the loss of over a decade of data and frozen stocks. I was able to write a report to the administration of the university that resulted in the hiring of an external scientific consultant and our lab was reimbursed for all of our destroyed equipment and supplies. 
 
\section{PUBLICATIONS}

    \textbf{Plourde SM}, Kravtsova N, Dawes AT (in preparation)  Mathematical and Biological Exploration of Cellular Patterning Based on Ground Truths.
    
    \textbf{Plourde SM}, Amom P, Tan M, Dawes AT, Dobritsa AA (2019) Changes in morphogen kinetics and pollen grain size are potential mechanisms of aberrant pollen aperture patterning in previously observed and novel mutants of Arabidopsis thaliana. PLOS Computational Biology 15(2): e1006800. \url{https://doi.org/10.1371/journal.pcbi.1006800}
    
    \textbf{Plourde SM}, Marin Z, Smith Z, Toner B, Batchelder K Khalil A. (2016). Computational growth model of breast microcalcification clusters in simulated mammographic environments. Computers in Biology and Medicine. 76. \url{https://doi.org/10.1016/j.compbiomed.2016.06.020}
          
    
          
\section{ORGANIZATIONS AND MEMBERSHIPS}
    MCDB Graduate Student Organization - Technology and Merchandise Chair - Aug 2018 to July 2019\\
    OSU Cycling team - Secretary 2018-2019, President 2019-2021, Vice President \& Social Media Manager 2021-present\\
    SIAM Graduate Student Membership

\end{resume} 
\end{document}





